


\documentclass[pdftex,11pt,a4paper]{report}

\usepackage{subfigure}

%%%% For reusing lengths as per http://tex.stackexchange.com/a/184992/13423
\makeatletter
\def\nnewlength#1{%
  \edef\reserved@a{\expandafter\@gobble\string#1} %
  \@ifundefined\reserved@a{%
    \edef\reserved@b{\expandafter\@carcube\reserved@a xxx\@nil} %
    \ifx\reserved@b\@qend %
      \typeout{  -- Not definable (1, \reserved@a): #1} %\@notdefinable %
    \else %
      \ifx\reserved@a\@qrelax %
        \typeout{  -- Not definable (2, \reserved@a): #1} %\@notdefinable %
      \else %
        \typeout{  -- Making newskip: #1}
        \newskip#1%
      \fi %
    \fi %
  }%
  {\typeout{  -- Not definable (E, \reserved@a): #1}} %\@notdefinable%
}
\makeatother




%%% For the perturbative optics diagram.
\usepackage{tikz}
\usetikzlibrary{arrows}
\usepackage{MnSymbol} % for circular arrows
\usepackage{rotating}
\usepackage{relsize}

\newcommand{\eps}{\varepsilon}

\newcommand{\leftpol}{\!\; \!\!\rcirclearrowright}
\newcommand{\rightpol}{\!\; \!\!\lcirclearrowright}
\newcommand{\epol}{\resizebox{0.8em}{1em}{$\lcirclearrowright$}}

\newcommand{\linpol}{\updownarrow}


\begin{document}


\begin{figure}
  \centering
  \subfigure[]{
    
\begingroup
\fontsize{10pt}{12pt}\selectfont

\begin{tikzpicture}[
   scale=0.5,
   level/.style={thick},
   photon/.style={thick,->,shorten >=0.5pt,shorten <=0.5pt,>=stealth},
 ]    
 \nnewlength{\smap} \setlength{\smap}{2cm} % small photon
 \nnewlength{\bigp} \setlength{\bigp}{3cm} % big photon
 \nnewlength{\sep} \setlength{\sep}{0.35cm} % (1/2) separation between the up and down arrows of each diagram
 \nnewlength{\lvlwidth} \setlength{\lvlwidth}{0.75cm} % (1/2) width of the horizontal 'level' lines
%
 \draw[level] (-\lvlwidth,  0cm) -- (\lvlwidth,  0cm);
 \draw[level] (-\lvlwidth,7\smap) -- (\lvlwidth,7\smap);
 \draw[photon] (-\sep, 0\smap) -- (-\sep, 1\smap) node[midway, left] {\begin{turn}{90}$\omega,\linpol$\end{turn}};
 \draw[photon] (-\sep, 1\smap) -- (-\sep, 2\smap) node[midway, left] {\begin{turn}{90}$\omega,\linpol$\end{turn}};
 \draw[photon] (-\sep, 2\smap) -- (-\sep, 3\smap) node[midway, left] {\begin{turn}{90}$\omega,\linpol$\end{turn}};
 \draw[photon] (-\sep, 3\smap) -- (-\sep, 4\smap) node[midway, left] {\begin{turn}{90}$\omega,\linpol$\end{turn}};
 \draw[photon] (-\sep, 4\smap) -- (-\sep, 5\smap) node[midway, left] {\begin{turn}{90}$\omega,\linpol$\end{turn}};
 \draw[photon] (-\sep, 5\smap) -- (-\sep, 6\smap) node[midway, left] {\begin{turn}{90}$\omega,\linpol$\end{turn}};
 \draw[photon] (-\sep, 6\smap) -- (-\sep, 7\smap) node[midway, left] {\begin{turn}{90}$\omega,\linpol$\end{turn}};
 \draw[photon] (\sep,7\smap) -- (\sep,  0cm) node[midway,right] {\begin{turn}{90}$7\omega,\linpol$\end{turn}};
%
\end{tikzpicture}

\endgroup





    \label{f8-initial-photon-diagrams-a}
  }
  \hspace{5mm}
  \subfigure[]{
    
\begingroup
\fontsize{10pt}{12pt}\selectfont

\begin{tikzpicture}[
   scale=0.5,
   level/.style={thick},
   photon/.style={thick,->,shorten >=0.5pt,shorten <=0.5pt,>=stealth},
 ]    
 \nnewlength{\smap} \setlength{\smap}{2cm} % small photon
 \nnewlength{\bigp} \setlength{\bigp}{3cm} % big photon
 \nnewlength{\sep} \setlength{\sep}{0.35cm} % (1/2) separation between the up and down arrows of each diagram
 \nnewlength{\lvlwidth} \setlength{\lvlwidth}{0.75cm} % (1/2) width of the horizontal 'level' lines
 \nnewlength{\cancelm} \setlength{\cancelm}{0.82cm} % horizontal offset of the center of the cancelling cross
 \nnewlength{\canceld} \setlength{\canceld}{0.75cm} % (1/2) horizontal width of the cancelling cross
 \nnewlength{\cancelv} \setlength{\cancelv}{1.7cm} % (1/2) height of the cancelling cross
%
 \draw[level] (-\lvlwidth,  0cm) -- (\lvlwidth,  0cm);
 \draw[level] (-\lvlwidth,7\smap) -- (\lvlwidth,7\smap);
 \draw[photon] (-\sep, 0\smap) -- (-\sep, 1\smap) node[midway, left] {\begin{turn}{90}$\omega,\rightpol$\end{turn}};
 \draw[photon] (-\sep, 1\smap) -- (-\sep, 2\smap) node[midway, left] {\begin{turn}{90}$\omega,\rightpol$\end{turn}};
 \draw[photon] (-\sep, 2\smap) -- (-\sep, 3\smap) node[midway, left] {\begin{turn}{90}$\omega,\rightpol$\end{turn}};
 \draw[photon] (-\sep, 3\smap) -- (-\sep, 4\smap) node[midway, left] {\begin{turn}{90}$\omega,\rightpol$\end{turn}};
 \draw[photon] (-\sep, 4\smap) -- (-\sep, 5\smap) node[midway, left] {\begin{turn}{90}$\omega,\rightpol$\end{turn}};
 \draw[photon] (-\sep, 5\smap) -- (-\sep, 6\smap) node[midway, left] {\begin{turn}{90}$\omega,\rightpol$\end{turn}};
 \draw[photon] (-\sep, 6\smap) -- (-\sep, 7\smap) node[midway, left] {\begin{turn}{90}$\omega,\rightpol$\end{turn}};
 \draw[photon] (\sep,7\smap) -- (\sep,  0cm) node[midway,right] {\begin{turn}{90}$7\omega,7\rightpol$\end{turn}};
%
 \draw[thick] (\cancelm - \canceld, 7\smap/2 - \cancelv) -- (\cancelm + \canceld, 7\smap/2 + \cancelv);
 \draw[thick] (\cancelm + \canceld, 7\smap/2 - \cancelv) -- (\cancelm - \canceld, 7\smap/2 + \cancelv);
%
\end{tikzpicture}

\endgroup





    \label{f8-initial-photon-diagrams-b}
  }
  \hspace{5mm}
  \subfigure[]{
    
\begingroup
\fontsize{10pt}{12pt}\selectfont

\begin{tikzpicture}[
   scale=0.5,
   level/.style={thick},
   photon/.style={thick,->,shorten >=0.5pt,shorten <=0.5pt,>=stealth},
 ]    
 \nnewlength{\smap} \setlength{\smap}{2cm} % small photon
 \nnewlength{\bigp} \setlength{\bigp}{3cm} % big photon
 \nnewlength{\sep} \setlength{\sep}{0.35cm} % (1/2) separation between the up and down arrows of each diagram
 \nnewlength{\lvlwidth} \setlength{\lvlwidth}{0.75cm} % (1/2) width of the horizontal 'level' lines
%
 \draw[level] (-\lvlwidth,  0cm) -- (\lvlwidth,  0cm);
 \draw[level] (-\lvlwidth,7\smap) -- (\lvlwidth,7\smap);
 \draw[photon] (-\sep, 0\smap) -- (-\sep, 1\smap) node[midway, left] {\begin{turn}{90} $\omega,\rightpol$\end{turn}};
 \draw[photon] (-\sep, 1\smap) -- (-\sep, 2\smap) node[midway, left] {\begin{turn}{90} $\omega,\rightpol$\end{turn}};
 \draw[photon] (-\sep, 2\smap) -- (-\sep, 3\smap) node[midway, left] {\begin{turn}{90} $\omega,\rightpol$\end{turn}};
 \draw[photon] (-\sep, 3\smap) -- (-\sep, 5\smap) node[midway, left] {\begin{turn}{90} $2\omega,\leftpol$\end{turn}};
 \draw[photon] (-\sep, 5\smap) -- (-\sep, 7\smap) node[midway, left] {\begin{turn}{90} $2\omega,\leftpol$\end{turn}};
 \draw[photon] (\sep,7\smap) -- (\sep,  0cm) node[midway,right] {\begin{turn}{90} $7\omega,\rightpol$\end{turn}};
%
\end{tikzpicture}

\endgroup





    \label{f8-initial-photon-diagrams-c}
  }
  \hspace{5mm}
  \subfigure[]{
    
\begingroup
\fontsize{10pt}{12pt}\selectfont

\begin{tikzpicture}[
   scale=0.5,
   level/.style={thick},
   photon/.style={thick,->,shorten >=0.5pt,shorten <=0.5pt,>=stealth},
 ]    
 \nnewlength{\smap} \setlength{\smap}{2cm} % small photon
 \nnewlength{\bigp} \setlength{\bigp}{3cm} % big photon
 \nnewlength{\sep} \setlength{\sep}{0.35cm} % (1/2) separation between the up and down arrows of each diagram
 \nnewlength{\lvlwidth} \setlength{\lvlwidth}{0.75cm} % (1/2) width of the horizontal 'level' lines
%
 \draw[level] (-\lvlwidth,  0cm) -- (\lvlwidth,  0cm);
 \draw[level] (-\lvlwidth,8\smap) -- (\lvlwidth,8\smap);
 \draw[photon] (-\sep, 0\smap) -- (-\sep, 1\smap) node[midway, left] {\begin{turn}{90} $\omega,\rightpol$\end{turn}};
 \draw[photon] (-\sep, 1\smap) -- (-\sep, 2\smap) node[midway, left] {\begin{turn}{90} $\omega,\rightpol$\end{turn}};
 \draw[photon] (-\sep, 2\smap) -- (-\sep, 4\smap) node[midway, left] {\begin{turn}{90} $2\omega,\leftpol$\end{turn}};
 \draw[photon] (-\sep, 4\smap) -- (-\sep, 6\smap) node[midway, left] {\begin{turn}{90} $2\omega,\leftpol$\end{turn}};
 \draw[photon] (-\sep, 6\smap) -- (-\sep, 8\smap) node[midway, left] {\begin{turn}{90} $2\omega,\leftpol$\end{turn}};
 \draw[photon] (\sep,8\smap) -- (\sep,  0cm) node[midway,right] {\begin{turn}{90} $8\omega,\leftpol$\end{turn}};
%
\end{tikzpicture}

\endgroup



    \label{f8-initial-photon-diagrams-d}
  }
  \hspace{5mm}
  \subfigure[]{
    
\begingroup
\fontsize{10pt}{12pt}\selectfont

\begin{tikzpicture}[
   scale=0.5,
   level/.style={thick},
   photon/.style={thick,->,shorten >=0.5pt,shorten <=0.5pt,>=stealth},
 ]    
 \nnewlength{\smap} \setlength{\smap}{2cm} % small photon
 \nnewlength{\bigp} \setlength{\bigp}{3cm} % big photon
 \nnewlength{\sep} \setlength{\sep}{0.35cm} % (1/2) separation between the up and down arrows of each diagram
 \nnewlength{\lvlwidth} \setlength{\lvlwidth}{0.75cm} % (1/2) width of the horizontal 'level' lines
%
 \draw[level] (-\lvlwidth,  0cm) -- (\lvlwidth,  0cm);
 \draw[level] (-\lvlwidth,8\smap) -- (\lvlwidth,8\smap);
 \draw[photon] (-\sep, 0\smap) -- (-\sep, 1\smap) node[midway, left] {\begin{turn}{90}$\omega,\linpol$\end{turn}};
 \draw[photon] (-\sep, 1\smap) -- (-\sep, 2\smap) node[midway, left] {\begin{turn}{90}$\omega,\linpol$\end{turn}};
 \draw[photon] (-\sep, 2\smap) -- (-\sep, 3\smap) node[midway, left] {\begin{turn}{90}$\omega,\linpol$\end{turn}};
 \draw[photon] (-\sep, 3\smap) -- (-\sep, 4\smap) node[midway, left] {\begin{turn}{90}$\omega,\linpol$\end{turn}};
 \draw[photon] (-\sep, 4\smap) -- (-\sep, 5\smap) node[midway, left] {\begin{turn}{90}$\omega,\linpol$\end{turn}};
 \draw[photon] (-\sep, 5\smap) -- (-\sep, 6\smap) node[midway, left] {\begin{turn}{90}$\omega,\linpol$\end{turn}};
 \draw[photon] (-\sep, 6\smap) -- (-\sep, 7\smap) node[midway, left] {\begin{turn}{90}$\omega,\linpol$\end{turn}};
 \draw[photon] (-\sep, 7\smap) -- (-\sep, 8\smap) node[midway, left] {\begin{turn}{90}$\omega,\linpol$\end{turn}};
 \draw[photon] (\sep, 8\smap) -- (\sep, 7\smap) node[midway, right] {\begin{turn}{90}$\omega,\linpol$\end{turn}};
 \draw[photon] (\sep,7\smap) -- (\sep,  0cm) node[midway,right] {\begin{turn}{90}$7\omega,\linpol$\end{turn}};
%
\end{tikzpicture}

\endgroup





    \label{f8-initial-photon-diagrams-e}
  }
  
  \caption{
  hello
  }
\label{f8-draft}
\end{figure}


\end{document}
